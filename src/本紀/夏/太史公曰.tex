太史公曰禹爲姒姓其後分封用國爲姓故有%
夏后氏有扈氏有男氏斟尋氏彤城氏襃氏費%
氏|[%
徐廣曰一云斟氏尋氏。索隱曰系本男作南尋作鄩費%
作弗而不云彤城及襃按周有彤伯蓋彤城氏之後張敖%
%
地理記云濟南平壽縣其地即古斟尋國%
又下云斟戈氏按左傳系本皆云斟灌氏%
]杞氏繒氏辛%
氏冥氏斟氏戈氏孔子正夏時學者多傳夏小%
正云|[%
禮運稱孔子曰我欲觀夏道是故之杞而不足徴也吾%
得夏時焉鄭玄曰得夏四時之書其存者有小正。索%
%
隱曰小正大戴記%
篇名正征二音%
]自虞夏時貢賦備矣或言禹會%
諸侯江南計功而崩因葬焉命曰會稽會稽者%
會計也|[%
皇覽曰禹冢在山隂縣會稽山上會稽山本名苗%
山在縣南去縣七里越傳曰禹到大越上苗山大%
%
會計爵有德封有功因而更名苗山曰會稽因病死葬𥯤棺%
穿壙深七尺上無瀉泄下無邸水檀髙三尺土堦三等周方%
%
一畝呂氏春秋曰禹葬會稽不煩人徒墨子曰禹葬會稽衣%
裘三領桐棺三寸地理志云山上有禹井禹祠相傳以爲下%
%
有羣鳥耘田也。索隱曰抵至也音丁礼反𥯤棺者以𥯤爲%
棺謂蘧蒢而斂非也禹雖儉約豈萬乗之主而臣子乃以蘧%
%
蒢裹尸乎墨子言桐棺三寸差近人情。正義曰括地志%
云禹陵在越州會稽縣南十三里廟在縣東南十一里%
]
