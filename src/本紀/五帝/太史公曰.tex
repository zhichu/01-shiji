太史公曰|[%
正義曰太史公司馬遷自謂也自叙傳云太史%
公曰先人有言又云太史公曰余聞之董生又%
%
云太史公遭李陵之禍明太史公司馬遷自號也遷爲太%
史公官題贊首也虞憙云古者主天官者皆上公非獨遷%
]%
學%
者多稱五帝尚矣|[%
索隱曰尚上也言乆逺%
也然尚矣文岀大戴禮%
]%
然尚書獨%
載堯以來而百家言黄帝其文不雅馴|[%
正義曰馴%
訓也謂百%
%
家之言皆非%
典雅之訓%
]%
薦紳先生難言之|[%
徐廣曰薦紳即縉%
紳也古字假借%
]%
孔子%
所傳宰予問五帝德及帝繫姓|[%
正義曰繫音奚計%
反五帝德及帝繫%
%
姓皆大戴禮文及孔子家語篇名漢儒者以%
二書非經恐不是聖人之言故或不傳學也%
]%
儒者或不傳%
|[%dup
索隱曰五帝德帝繫姓皆大戴禮及孔子家語篇名以二者%
皆非正經故漢時儒者以爲非聖人之言故多不傳學也%
]%
余甞西至空峒|[%
正義曰余太史公自稱也甞曾也空桐%
山在原州平髙縣西百里黄帝問道於%
%
廣城%成
子處%
]%
北過涿鹿|[%
正義曰涿鹿山在嬀州東南五十里%
山側有涿鹿城即黄帝堯舜之都也%
]%
東%
漸於海南浮江淮矣至長老皆各往往稱黄帝%
堯舜之處風教固殊焉緫之不離古文者近是%
|[%
索隱曰古文即帝德帝系%
二書也近是聖人之說%
]%
予觀春秋國語其發明五%
帝德帝繫姓章矣|[%
索隱曰太史公言己以春秋國語%
古書博加考驗益以發明五帝德%
%
等說甚%
章著也%
]%
顧弟弗深考|[%
徐廣曰弟但也史說漢書見此者%
非一又左思蜀都賦曰弟如滇池%
%
而不詳者多以爲字誤學者安可不博觀乎。正義曰顧念%
也弟且也太史公言博考古文擇其言表見之不虛甚章著%
%
矣思念亦且不%
須更深考論%
]%
其所表見皆不虛|[%
索隱曰言帝德帝系%
所有表見者皆不爲%
%
虛妄%
也%
]%
書缺有間矣|[%閒
正義曰言古文尚書缺失其%
間多矣而無說黄帝之語%閒
]%
其軼%
乃時時見於他說|[%
索隱曰言古典殘缺有年載故曰有%
閒然皇帝遺事散軼乃時時旁見於%
%
他記說即帝德帝系等說也故己%
今採按而備論黄帝已來事耳%
]%
非好學深思心知其%
意固難爲淺見寡聞道也余并論次擇其言尤%
雅者故著爲本紀書首|[%
正義曰太史公據古文并諸%
子百家論次擇其言語典雅%
%
者故著爲五帝本紀在%
史記百三十篇書之首%
]
