黃帝者|[%
徐廣曰號有熊。索隱曰按有土德之瑞土色黃%
故稱黃帝猶神農火德王而稱炎帝然也此以黃%
%
帝爲五帝之首盖依大戴禮五帝德又譙周宋均亦以爲然%
而孔安國皇甫謐帝王代紀及孫氏注系本並以伏犧神農%
%
黃帝爲三皇少昊高陽高辛唐虞爲五帝注號有熊者以其%
本是有熊國君之子故也都軒轅之丘因以爲名又以爲號%
%
又據左傳亦號帝鴻氏也。正義曰輿地志云涿鹿本名彭%
城黃帝初都遷有熊也按黃帝有熊國君乃少典國君之次%
%
子號曰有熊氏又曰縉雲氏又曰帝鴻氏亦曰帝軒氏母曰%
附寶之祁野見大電繞北斗樞星感而懷孕二十四月而生%
%
黃帝於壽丘壽丘在魯東門之北今在兗州曲阜縣東北六%
里生日角龍顏有景雲之瑞以土德王故曰黃帝封泰山禪%
%
亭亭在%
牟隂%
]少典之子|[%
譙周曰有熊國君少典之子也皇甫謐%
曰有熊今河南新鄭是也。索隱曰少%
%
典者諸侯國號非人名也又按國語云少典娶有蟜氏女而%
生炎帝然則炎帝亦少典之子炎黃二帝雖則承帝王代紀%
%
中間凡隔八帝五百餘年若以少典是其父名豈黃帝經五%
百餘年而始代炎帝後爲天子乎何其年之長也又按秦本%
%
紀云顓頊氏之裔孫曰女脩吞玄鳥之卵而生大業大業娶%
少典氏而生柏翳明少典是國號非人名也黃帝即少典氏%
%
後代之子孫賈逵亦以左傳高陽氏有才子八人亦謂其後%
代子孫而稱爲子是也譙周字允南蜀人魏散騎常侍徵不%
%
拜此注所引者是其人所著古史考之說也皇甫謐字士%
安晉人號玄晏先生今所引者是其所作帝王代紀也%
]姓%
%
公孫名曰軒轅|[%
索隱曰按皇甫謐云黃帝生於壽丘長%
於姬水因以爲姓居軒轅之丘因以爲%
%
名又以爲號是本姓公%
孫長居姬水因改姓姬%
]生而神靈弱而能言|[%
索隱曰%
弱謂幼%
%
弱時也盖未合能言之時而黃帝即言所以爲神異也潘岳%
有哀弱子篇其子未七旬曰弱。正義曰言神異也易曰隂%
%
陽不測之謂神書云人惟%
萬物之靈故謂之神靈也%
]幼而徇齊|[%
徐廣曰墨子曰年踰%
十五則聦明心慮無%
%
不徇通矣駰案徇疾齊速也言聖德幼而疾速也。索隱曰%
斯文未明今案徇齊皆德也書曰聦明齊聖左傳曰子雖齊%
%
聖齊謂聖德齊肅又按孔子家語及大戴禮並作叡齊一本%
作慧齊叡慧皆智也太史公採大戴禮而爲此紀今彼文無%
%
作徇者史記舊本亦有作濬齊盖古字假借徇爲濬濬深也%
義亦並通爾雅齊速俱訓爲疾尚書大傳曰多聞而齊給鄭%
%
注云齊疾也今裴氏注云徇亦訓疾未見所出或當讀徇爲%
迅迅於爾雅與齊俱訓疾則迅濬雖異字而音同也又爾雅%
%
曰宣徇遍也濬通也是遍之與通義亦相近言黃帝幼而才%
智周徧且辯給也故墨子亦云年踰五十則聦明心慮不徇%
%
通矣俗本作十五非是按謂年%
老踰五十不聦明何得云十五%
]長而敦敏成而聦明|[%
正%
義%
%
曰成謂年二十冠成人也聦明聞見%
明辯也此以上至軒轅皆大戴禮文%
]軒轅之時神農氏%
%
世衰|[%
皇甫謐曰易稱庖犧氏沒神農氏作是爲炎帝班固%
曰教民耕農故號曰神農。索隱曰世衰謂神農氏%
%
後代子孫道德衰薄非指炎帝之身即班固所謂叅虛皇甫%
謐所云帝榆罔是也。正義曰帝王世紀云神農氏姜姓也%
%
母曰任姒有蟜氏女登爲少典妃遊華陽有神龍首感生炎%
帝人身牛首長於姜水有聖德以火德王故號炎帝初都陳%
%
又徙魯又曰魁隗氏又曰連山氏又曰列山氏括地志云厲%
山在隨州隨縣北百里山東有石穴曰神農生於厲鄉所謂%
%
列山氏也春%
秋時爲厲國%
]諸侯相侵伐暴虐百姓而神農氏弗%
%
能征於是軒轅乃習用干戈㠯征不享|[%
索隱曰謂%
用干戈以%
%
征諸侯之不朝享者本或作亭%
亭訓直以征諸侯之不直者%
]諸侯咸來賔從而蚩%
%
尤最爲暴莫能伐|[%
應劭曰蚩尤古天子瓉曰孔子三%
朝記曰蚩尤庶人之貪者。索隱%
%
曰按此紀云諸侯相侵伐蚩尤最爲暴則蚩尤非爲天子也%
又管子曰蚩尤受盧山之金而作五兵明非庶人盖諸侯號%
%
也劉向別錄云孔子見魯哀公問政比三朝退而爲此記故%
曰三朝凡七篇並入大戴禮今此文見用兵篇也。正義曰%
%
龍魚河圖云黃帝攝政有蚩尤兄弟八十一人並獸身人語%
銅鐵額食沙造五兵仗刀戟大弩威振天下誅殺無道萬民%
%
欽命黃帝行天子事黃帝以仁義不能禁止蚩尤乃仰天而%
歎天遣玄女下授黃帝兵符伏蚩尤後天下復擾亂黃帝遂%
%
畫蚩尤形像以威天下咸謂蚩尤不死八方皆爲殄滅山海%
經云黃帝令應龍攻蚩尤蚩尤請風伯雨師以從大風雨黃%
%
帝乃下天女曰魃以止雨雨止遂殺%
蚩尤孔安國曰九黎君號蚩尤是也%
]炎帝欲侵陵諸侯%
%
諸侯咸歸軒轅軒轅乃修德振兵|[%
正義曰%
振整也%
]治五%
%
氣|[%
王肅曰五行之氣。索隱曰謂春甲%
乙木氣夏丙丁火氣之屬是五氣也%
]藝五種|[%
藝樹也%
詩云藝%
%
之荏菽周禮曰榖宜五種鄭玄曰五種黍稷菽麥稻也。索%
隱曰藝音蓺藝種也樹也五種即五穀也音朱用反此注所%
%
引見詩大雅生民之篇爾雅云荏菽戎也郭璞曰今之胡%
豆鄭氏曰豆之大者是也。正義曰藝音魚曳反種音腫%
]撫%
%
萬民度四方|[%
王肅曰度四方而安撫之%
。正義曰度音徒洛反%
]教熊羆貔貅%
%
貙虎|[%
索隱曰書云如虎如貔爾雅云貔白狐禮曰前有摯%
獸則載貔貅是也爾雅又曰貙獌似貍此六者猛獸%
%
可以教戰周禮有服不氏掌教擾猛獸即古服牛乘馬亦其%
類也。正義曰熊音雄羆音碑貔音毗貅音休貙音丑于反%
%
羆如熊黃白色郭璞云貔執夷虎屬也按言%
教士卒習戰以猛獸之名名之用威敵也%
]以與炎帝戰%
%
於阪泉之野|[%
服虔曰阪泉地名皇甫謐曰在上谷。正義%
曰阪音白板反括地志云阪泉今名黃帝泉%
%
在嬀州懷戎縣東五十六里出五里至涿鹿東北與涿水合%
又有涿鹿故城在嬀州東南五十里本黃帝所都也晉太康%
%
地里志云涿鹿城東一里有阪泉上有%
黃帝祠按阪泉之野則平野之地也%
]三戰然後得其%
%
志|[%
正義曰謂黃帝%
克炎帝之後%
]蚩尤作亂不用帝命|[%
正義曰言蚩%
尤不用黃帝%
%
之命%
也%
]於是黃帝乃徵師諸侯與蚩尤戰於涿鹿%
%
之野|[%
服虔曰涿鹿山名在涿郡張晏曰涿鹿在上谷。索%
隱曰或作濁鹿古今字異耳按地理志上谷有涿鹿%
%
縣然則服虔云%
在涿郡者誤也%
]遂禽殺蚩尤|[%
皇覽曰蚩尤冢在東平郡%
壽張縣闞鄉城中高七丈%
%
民常十月祀之有赤氣出如匹絳帛民名爲蚩尤旗肩髀冢%
在山陽郡鉅野縣重聚大小與闞冢等傳言黃帝與蚩尤戰%
%
於涿鹿之野黃帝殺之身體異處故別葬之。索隱曰按皇%
甫謐云黃帝使應龍殺蚩尤于凶黎之谷或曰黃帝斬蚩尤%
%
于中兾因名其地曰絕轡之野皇覽書名也記先代冢墓之%
處宜皇王之省覽故日皇覽是魏人王象繆襲等所撰也%
%
]而諸侯咸尊軒轅爲天子伐神農氏是爲黃帝%
%
天下有不順者黃帝從而征之平者去之|[%
正義%
曰平%
%
服者即%
去也%
]披山通道|[%
徐廣曰披他本亦作陂字盖當爲詖詖%
者旁其邊之謂也披語誠合今世然古%
%
今不必同也。索隱曰披音如字謂披山林%
草木而行以通道也徐廣音詖恐稍紆也%
]未嘗寧居東%
%
至于海登丸山|[%
徐廣曰丸一作凡駰案地理志曰丸山%
在郎耶朱虛縣。索隱曰凢音扶嚴反%
%
。正義曰丸音桓括地志云丸山即丹山在青州臨朐縣界%
朱虛故縣西北二十里丹水出焉凡音紈守節括地志唯有%
%
凡山蓋凡山丸山是一山耳諸處字誤或丸或凡也漢書郊%
祀志云禪丸山顏師古云在朱虛亦與括地志相合明丸山%
是%
也%
]及岱宗|[%
正義曰泰山東岳也在兗%
州博城縣西北三十里也%
]西至于空桐|[%
應%
劭%
%
曰山名韋昭%
曰在隴右%
]登雞頭|[%
索隱曰山名也後漢王孟塞雞頭道%
在隴西一曰崆峒山之別名。正義%
%
曰括地志云空桐山在肅州祿福縣東南六十里抱朴子內%
篇云黃帝西見中黃子受九品之方過空桐從廣成子受自%
%
然之經即此山括地志又云笄頭山一名崆峒山在原州平%
陽縣西百里禹貢涇水所出輿地志云或即雞頭山也酈元%
%
云蓋大隴山異名也莊子云廣成子學道崆峒山黃帝問道%
於廣成子蓋在此按二處崆峒皆云黃帝登之未詳孰是%
%
]南至于江登熊湘|[%
封禪書曰南伐至于召陵登熊山地%
理志曰湘山在長沙益陽縣。正義%
%
曰括地志云熊耳山在商州洛縣西十里齊桓公登之以%
望江漢也湘山一名艑山在岳州巴陵縣南十八里也%
]北%
%
逐葷粥|[%
匈奴傳曰唐虞以上有山戎獫狁葷粥居于北蠻%
。索隱曰匈奴別名也唐虞已上曰山戎亦曰薰%
%
粥夏曰淳維殷曰鬼方周曰玁狁漢%
曰匈奴。正義曰葷音薰粥音育%
]合符釡山|[%
索隱曰合%
諸侯符契%
%
圭瑞而朝之於釡山猶禹會諸侯於塗山然也又按郭子橫%
洞冥記稱東方朔云東海大明之墟有釡山山出瑞雲應王%
%
者之符命如堯時有赤雲之祥之類盖黃帝黃雲之瑞故曰%
合符應於釡山也。正義曰括地志云釡山在嬀州懷戎縣%
%
北三里山%
上有舜廟%
]而邑于涿鹿之阿|[%
正義曰廣平曰阿涿鹿山%
名已見上涿鹿故城在山%
%
下即黃帝所都之%
邑於山下平地%
]遷徙往來無常處以師兵爲營%
%
衞|[%
正義曰環繞軍兵爲營以%
自衞若轅門即其遺象%
]官名皆以雲命爲雲師|[%
%
應劭曰黃帝受命有雲瑞故以雲紀事也春官爲青雲夏官%
爲縉雲秋官爲白雲冬官爲黑雲中官爲黃雲張晏曰黃帝%
%
有景雲之應因%
以名師與官%
]置左右大監監于萬國|[%
正義曰監上%
監去聲下監%
%
平聲若周%
邵分陝也%
]萬國和而鬼神山川封禪與爲多焉|[%
徐%
廣%
%
曰多一作朋。索隱曰與音羊汝反與猶許也言萬國和同%
而鬼神山川封禪祭祀之事自古以來帝皇之中推許黃帝%
%
以爲多多%
猶大也%
]獲寶鼎迎日推策|[%
晉灼曰策數也迎數之%
也瓉曰日月朔望未來%
%
而推之故曰迎日。索隱曰封禪書曰黃帝得寶鼎神策下%
云於是推策迎日則神策者神蓍也黃帝得蓍以推筭歷數%
%
於是逆知節氣日辰之將來故曰推策迎日也。正義曰%
筴音策迎逆也黃帝受神筴大撓造甲子容成造曆是也%
]舉%
%
風后力牧常先大鴻|[%
鄭玄曰風后黃帝三公也班固%
曰力牧黃帝相也。正義曰舉%
%
任用四人皆帝臣也帝王世紀云黃帝夢大風吹天下之塵%
垢皆去又夢人執千鈞之弩驅羊萬羣帝寤而嘆曰風爲號%
%
令執政者也垢去土后在也天下豈有姓風名后者哉夫千%
鈞之弩異力者也驅羊數萬羣能牧民爲善者也天下豈有%
%
姓力名牧者也於是依二占而求之得風后於海隅登以爲%
相得力牧於大澤進以爲將黃帝因著占夢經十一卷藝文%
%
志云風后兵法十三篇圖三卷孤虛二十卷力牧兵法十五%
篇鄭玄云風后黃帝之三公也按黃帝仰天地置列侯衆官%
%
以風后配上台天老配中台五聖配下台謂之三公也封禪%
書云鬼臾區號大鴻黃帝大臣也死葬雍故鴻冢是藝文志%
%
云鬼容區兵%
法三篇也%
]以治民順天地之紀|[%
正義曰言黃帝順天%
地隂陽四時之紀也%
%
]幽明之占|[%
正義曰幽隂明陽也占數也言隂陽五%
行黃帝占數而知之此文見大戴禮%
]死生%
%
之說|[%
徐廣曰一云幽明之數合死生之說。正義曰%
說謂儀制也民之生死此謂作儀制禮則之說%
]存亡%
%
之難|[%
索隱曰存亡猶安危也易曰危者安其位亡者保其%
存是也難猶說也凢事是非未盡假以往來之詞則%
%
曰難又上文有死生之說故此云存亡之難所以韓非著書%
有說林說難也。正義曰難音乃憚反存亡猶生死也黃帝%
%
之前未有衣裳屋宇及黃帝造屋宇%
制衣服營殯葬萬民故免存亡之難%
]時播百穀草木|[%
王%
肅%
%
曰時是也。索隱曰爲一句。正義曰%
言順四時之所置而布種百穀草木也%
]淳化鳥獸蟲蛾|[%
%
索隱曰爲一句蛾音牛綺反一作豸豸言淳化廣被及之。%
正義曰蛾音魚起反又音豸豸音直氏反蟻蚍蜉也爾雅曰%
%
有足曰虫%
無足曰豸%
]旁羅日月星辰水波土石金玉|[%
徐廣曰%
波一作%
%
沃。索隱曰旁非一方羅廣布也今按大戴禮作歷離離即%
羅也言帝德旁羅日月星辰水波及至土石金玉謂日月揚%
%
光海水不波山不藏珎皆是帝德廣被也。正義曰旁羅猶%
遍布也日月隂陽時節也星二十八宿也辰日月所會也水%
%
波瀾漪也言天不異災土無%
別害水少波浪山出珍寶%
]勞勤心力耳目節用水火%
%
材物|[%
正義曰節時節也水陂障決洩也火山野禁放也材%
木也物事也言黃帝教民江湖陂澤山林原隰皆收%
%
採禁捕以時用之有節令得其利也大戴禮云宰我問於孔%
子曰予聞榮伊曰黃帝三百年請問黃帝者人耶何以至三%
%
百年孔子曰勞勤心力耳目節用水火林物生而民得其利%
百年死而民畏其神百年亡而民用其教百年故曰三百年%
%
也%
]有土德之瑞故號黃帝|[%
索隱曰炎帝火黃帝土代%
之即黃龍地螾見是也螾%
%
土精大五六圍長十餘丈螾%
音引。正義曰螾音以刃反%
]黃帝二十五子其得姓者%
%
十四人|[%
索隱曰舊解破四爲三言得姓十三人耳今按國%
語胥臣云黃帝之子二十五宗其得姓者十四人%
%
爲十二姓姬酉祁己滕葴任荀僖姞嬛依是也唯青陽與夷%
鼓同己姓又云青陽與蒼林爲姬姓上則十四人爲十二姓%
%
其文甚明唯姬姓再稱青陽與蒼林盖國語文誤所以致令%
前儒共疑其姬姓青陽當爲玄囂是帝嚳祖本與黃帝同姬%
%
姓其國語上文青陽即是少昊金天氏爲己姓者耳旣理在%
不疑無煩破四爲三。正義曰僖音力其反姞其吉反嬛音%
%
在宣%
反%
]黃帝居軒轅之丘|[%
皇甫謐曰受國於有熊居軒轅%
之丘故因以爲名又以爲號山%
%
海經曰在窮山之際西射之南張%
晏曰作軒冕之服故謂之軒轅%
]而娶於西陵之女|[%
正%
義%
%
曰西陵%
國名也%
]是爲嫘祖|[%
徐廣曰祖一作俎嫘力追反。索隱曰%
一曰雷祖音力堆反。正義曰一作傫%
%
]嫘祖爲黃帝正妃|[%
索隱曰按黃帝立四妃象后妃四星%
皇甫謐云元妃西陵氏女曰累祖生%
%
昌意次妃方雷氏女曰女節生青陽次妃彤魚氏女生夷鼓%
一名蒼林次妃嫫母班在三人之下按國語夷鼓蒼林是二%
%
人又按漢書古今人表彤魚氏生%
夷鼓嫫母生蒼林不得如謐所說%
]生二子其後皆有天%
%
下其一曰玄囂是爲青陽|[%
太史公曰據大戴禮以累%
祖生昌意及玄囂玄囂即%
%
青陽也皇甫謐以青陽爲少昊乃方雷氏所生是其所見異%
也。索隱曰玄囂帝嚳之祖按皇甫謐及宋衷皆云玄囂青%
%
陽即少昊也今此紀下云玄囂不得在帝位則太史公意青%
陽非少昊明矣而此又云玄囂是爲青陽當是誤也謂二人%
%
皆黃帝子並列其名所以前史因誤以玄囂青陽爲一人耳%
宋衷又云玄囂青陽是爲少昊繼黃帝立者而史不叙盖少%
%
昊金德王非五運之次%
故叙五帝不數之也%
]青陽降居江水|[%
正義曰括地志%
云安陽故城在%
%
豫州新恩縣西南八十里應劭云古%
江國也地理志亦云安陽古江國也%
]其二曰昌意降居%
%
若水|[%
索隱曰降下也言帝子爲諸侯降居江水江水若水%
皆在蜀即所封國也水經曰水出旄牛徼外東南至%
%
故關爲若水南過邛都又東北至朱%
提縣爲盧江水是蜀有此二水也%
]昌意娶蜀山氏女%
%
曰昌僕生高陽高陽有聖德焉|[%
正義曰華陽國志%
及十三州志云蜀%
%
之先肇於人皇之際黃帝爲子昌意娶蜀山氏後子孫因封%
焉帝顓頊高陽氏黃帝之孫昌意之子母曰昌僕亦謂之女%
%
樞河圖云瑤光如蜺貫月正白感女樞於%
幽房之宮生顓頊首戴干戈有德文也%
]黃帝崩|[%
皇甫謐%
曰在位%
%
百年而崩年百一十一歳。索隱曰按大戴禮宰我問孔子%
曰榮伊言黃帝三百年請問黃帝何人也抑非人也何以至%
%
三百年乎對曰生而人得其利百年死而人畏其神百年亡%
而人用其教百年則士安之說略可憑矣。正義曰列仙傳%
%
云軒轅自擇亡日與羣臣辭還葬%
橋山山崩棺空唯有劔舄在棺焉%
]葬橋山|[%
皇覽曰黃帝冢%
在上郡橋山。%
%
索隱曰地理志橋山在上郡同陽縣山有黃帝冢也。正義%
曰括地志云黃帝陵在寧州羅川縣東八十里子午山地理%
%
志云上郡陽周縣橋山南有黃帝冢按陽%
周隋改爲羅川爾雅云山銳而高曰橋也%
]其孫昌意之%
%
子高陽立是爲帝顓頊也%
