太史公曰|[%
一]正義曰太史公司馬遷自謂也自敍傳云太史公曰先人有言又云太史公曰余聞之董生又云太史公遭李陵之禍明太史公司馬遷自號也遷爲太史公官題贊首也虞憙云古者主天官者皆上公非獨遷學者多稱五帝尚矣|[%
二]索隱曰尚上也言久逺也然尚矣文岀大戴禮然尚書獨載堯以來;而百家言黃帝其文不雅馴|[%
三]正義曰馴訓也謂百家之言皆非典雅之訓薦紳先生難言之|[%
四]徐廣曰薦紳即縉紳也古字假借孔子所傳宰予問五帝德及帝繫姓|[%
五]正義曰繫音奚計反儒者或不傳|[%
六]索隱曰五帝德帝繫姓皆大戴禮及孔子家語篇名以二者皆非正經故漢時儒者以爲非聖人之言故多不傳學也余甞西至空桐|[%
七]正義曰余太史公自稱也甞曾也空桐山在原州平髙縣西百里黃帝問道於廣成子處北過涿鹿|[%
八]正義曰涿鹿山在嬀州東南五十里山側有涿鹿城即黃帝堯舜之都也東漸於海南浮江淮矣至長老皆各徃徃稱黃帝堯舜之處風教固殊焉總之不離古文者近是|[%
九]索隱曰古文即帝德帝系二書也近是聖人之說予觀春秋國語其發明五帝德帝繫姓章矣|[%
一0]索隱曰太史公言己以春秋國語古書博加考驗益以發明五帝德等說甚章著也顧弟弗深考|[%
一一]徐廣曰弟但也史記漢書見此者非一又左思蜀都賦曰弟如滇池而不詳者多以爲字誤學者安可不博觀乎正義曰顧念也弟且也太史公言博考古文擇其言表見之不虛甚章著矣思念亦且不須更深考論其所表見皆不虛|[%
一二]索隱曰言帝德帝系所有表見者皆不爲虛妄也書缺有閒矣|[%
一三]正義曰言古文尚書缺失其閒多矣而無說黃帝之語其軼乃時時見於他說|[%
一四]索隱曰言古典殘缺有年載故曰有閒然皇帝遺事散軼乃時時旁見於他記說即帝德帝系等說也故己今採案而備論黃帝已來事耳非好學深思心知其意固難爲淺見寡聞道也余并論次擇其言尤雅者故著爲本紀書首|[%
一五]正義曰太史公據古文并諸子百家論次擇其言語典雅者故著爲五帝本紀在史記百三十篇書之首

48
【索隱述贊】帝岀少典居于軒丘旣代炎曆遂禽蚩尤髙陽嗣位靜深有謀小大逺近莫不懷柔爰洎帝嚳列聖同休帝摯之弟其號放勳就之如日望之如雲郁夷東作昧谷西曛明敭庂陋玄德升聞能讓天下賢哉二君

